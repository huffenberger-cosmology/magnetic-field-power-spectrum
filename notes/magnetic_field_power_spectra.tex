\documentclass{article}

\usepackage{fullpage}
\usepackage{natbib}
\usepackage{aas_macros}
%\usepackage{hyperref}

\begin{document}
\title{Magnetic Field Power Spectrum}
\author{Kevin Huffenberger}
\date{31 October 2019}
\maketitle
 
\section{Basic definitions}

Following \citet{2002PhRvD..65l3004M}, the power spectrum of a homogeneous isotropic magnetic field is
\begin{equation}
  \langle B_i(\mathbf{k})  B_j(\mathbf{k}')^* \rangle = (2\pi)^3 P_{ij}(\mathbf{k}) P(k) \delta(\mathbf{k - k}')
  \label{eqn:psdef}
 \end{equation}
 and the projection operator to the transverse plane is
 \begin{equation}
    P_{ij}(\mathbf{k})  = \delta_{ij} - \hat k_i \hat k_j
 \end{equation}
 and the $\hat k_i$ are the components of the unit vector in the longitudinal ($\mathbf{k}$) direction.  The projection operator is by definition symmetric, and has some useful properties.  First, it projects every longitudinal vector to zero,
 \begin{equation}
   P_{ij} \hat k_j =  \hat k_i P_{ij} = 0,
   \label{eqn:transproj}
 \end{equation}
 and second, it is idempotent, meaning its own square,
 \begin{equation}
   P_{ij}P_{jk} = P_{ik}.
   \label{eqn:idemproj}
 \end{equation}
  The projection operator is the same for opposite wavenumbers:
 \begin{equation}
   P_{ij}(-\mathbf{k}) = \delta_{ij} - (-k_i)(-k_j) = P_{ij}(\mathbf{k})
 \end{equation}

 
 If there is a random vector field with specified power spectrum for each independent component:
 \begin{equation}
   \langle X_i(\mathbf{k}) X_j(\mathbf{k}') \rangle = (2\pi)^3 P(k) \delta_{ij}\delta(\mathbf{k - k'})
 \end{equation}
 Let the magnetic field be 
 \begin{equation}
   B_i(\mathbf{k}) = P_{ij}(\mathbf{k}) X_j(\mathbf{k})
 \end{equation}
 and then the power spectrum of the magnetic modes is
 \begin{eqnarray}
   \langle B_i(\mathbf{k})  B_j(\mathbf{k}')^* \rangle &=&  P_{ij}(\mathbf{k}) \langle X_j(\mathbf{k}) X_l(\mathbf{k'}) \rangle P_{ml}(\mathbf{k'}) \\ \nonumber
   &=& (2\pi)^3 P_{ij}(\mathbf{k}) P_{ml}(\mathbf{k'}) P(k) \delta_{jl} \delta(\mathbf{k - k'})\\ \nonumber
   &=& (2\pi)^3 P_{ij}(\mathbf{k}) P(k) \delta(\mathbf{k - k}')
 \end{eqnarray}
\section{FFT conventions}
 
 The standard Fourier transform convention is approximated by a Riemann sum:
 \begin{equation}
   B_i(\mathbf{k}) = \int d^3x \exp(-i \mathbf{k \cdot x}) B_i(\mathbf{x}) = \Delta_{\rm vol} \sum_{\rm{index}\ \mathbf{n}} \exp(-2\pi i \mathbf{p}\cdot (\mathbf{n/N}) ) B_i(\mathbf{x_n})
 \end{equation}
 The expression following the sum is a standard 3-dimensional FFT, and
 \begin{equation}
   \Delta_i = L_i/N_i
 \end{equation}
 is the pixel size in terms of the box size and number of pixels, with $\Delta_{\rm vol} = \prod_i \Delta_i$ is the pixel volume.  The symbol $(\mathbf{n/N})$ is shorthand for the vector with component $n_i/N_i$.
The position in real space is 
 \begin{equation}
   x_n,i = \Delta_i n_i 
 \end{equation}
 while the position in Fourier space is
 \begin{equation}
   k_{p,i} = p_i \Delta_{k,i} = 2\pi p_i/N_i \Delta_i
 \end{equation}
 where
 \begin{equation}
   \Delta_{k,i} = 2\pi/N_i\Delta_i
   \end{equation}
is the pixel size in Fourier space and $\Delta_{{\rm vol}\,k} = \prod_i \Delta_{k,i}$ is the volume of a Fourier pixel.

Similarly, the inverse Fourier transform is approximated as
\begin{equation}
   B_i(\mathbf{x}) = \int \frac{d^3k}{(2\pi)^3} \exp(i \mathbf{k \cdot x}) B_i(\mathbf{k}) = \frac{\Delta_{{\rm vol}\, k}}{(2\pi)^3} \sum_{\rm{index}\ \mathbf{p}} \exp(2\pi i \mathbf{p}\cdot (\mathbf{n/N}) ) B_i(\mathbf{k_p})
 \end{equation}
 and the sum is a standard inverse FFT.  (Note: $\Delta_{{\rm vol}}\Delta_{{\rm vol}\,k}/(2\pi)^3 = 1/N$, and counters the factor $N$ renormalization that occurs from successive forward-backward FFTs.)


On the discrete grid, the harmonic space $\delta$-function is approximated at $\delta(\mathbf{k-k'}) = \delta_\mathbf{pp'} / \Delta_{{\rm vol}\, k}$.  (It integrates to unity over that single pixel.)

\section{Generating a stochastic field with specified spectrum}

At every Fourier pixel $\mathbf{p}$, generate a 3-vector of independent, complex Gaussian deviates with unit variance
\begin{eqnarray}
  {\rm Re}(g_i(\mathbf{p})) \sim {\cal N}(\mu = 0, \sigma^2 = 1/2) \\ \nonumber
  {\rm Im}(g_i(\mathbf{p})) \sim {\cal N}(\mu = 0, \sigma^2 = 1/2)  
\end{eqnarray}
(We have to be careful for the $\mathbf{p} = 0$ mode which has no imaginary part, so the real part needs unit variance.)

Thus the covariance of these independent deviates is
\begin{equation}
  \langle g_i(\mathbf{p}) g_j^*(\mathbf{p}') \rangle = \delta_{ij} \delta_\mathbf{pp'}
\end{equation}
Then we assign the magnetic field modes as
\begin{equation}
  B_i (\mathbf{k_p}) = \left[ \frac{(2\pi)^3 P(k)}{\Delta_{{\rm vol}\,k}} \right]^{1/2} P_{il}(\mathbf{k_p}) g_l(\mathbf{p})
\end{equation}
Thus the covariance of modes is what we expect on the discrete grid:
\begin{eqnarray}
  \langle B_i(\mathbf{k_p}) B^*_j(\mathbf{k_{p'}}) \rangle &=&   \left[ \frac{(2\pi)^3 P(k)}{\Delta_{{\rm vol}\,k}} \right] P_{il}(\mathbf{k_p}) \left\langle g_l(p) g^*_m(\mathbf{p'}) \right\rangle P_{mj}(\mathbf{k_{p'}}) \\ \nonumber
  &=&   {(2\pi)^3 P(k)}  P_{il}(\mathbf{k_p}) \delta_{lm} \delta_{\mathbf{pp}'} P_{lm}(\mathbf{k_p})/{\Delta_{{\rm vol}\,k}} \\ \nonumber
  &=&  {(2\pi)^3 P(k)}  P_{ij}(\mathbf{k_p}) \delta_{\mathbf{pp}'}/{\Delta_{{\rm vol}\,k}}
\end{eqnarray}
where the last equality uses the idempotent property from Equation~\ref{eqn:idemproj}.
with the Kronecker $\delta$ over the pixel volume standing in for the Dirac delta function from Equation~\ref{eqn:psdef}.

We can test the divergence of the magnetic field in Fourier space, and find that it is zero:
\begin{equation}
  k_i B_i \propto k_i P_{il} g_l = k \hat k_i P_{il} g_l = 0.
\end{equation}
by Equation \ref{eqn:transproj}.

Thus we have shown that this definition for the magnetic field modes is both divergence free and has the specified power spectrum.  We obtain the real-space field via an inverse Fourier transform.

\section{Evaluating the power spectrum}

The projection operator $P_{ij}$ eliminates longitudinal modes and so does not in general have an inverse.

We can make use of the following fact to make an estimate of the power spectrum.
\begin{equation}
P_{ji} \langle B_i(\mathbf{k_p}) B^*_j(\mathbf{k_{p}}) \rangle =  {(2\pi)^3 P(k)}  P_{jj}(\mathbf{k_p})/{\Delta_{{\rm vol}\,k}}
\end{equation}


We can make an estimate for the power spectrum P(k) by contracting Fourier modes with the projection operator and dividing out the trace of the projection.
\begin{equation}
  \bar P(k_1 <k < k_2) = \frac{\Delta_{{\rm vol}\,k}}{(2\pi)^3 }  \frac{\sum_{ \mathbf{p}} P_{ji}(\mathbf{k_p}) B_i(\mathbf{k_p}) B^*_j(\mathbf{k_p}) }{\sum_{ \mathbf{p}} P_{jj}(\mathbf{k_p)}}
\end{equation}
where the sum is over Fourier pixels $\mathbf{p}$ within the band  ($ k_1 < |\mathbf{k_p}|<k_2$).  It is normalized by the trace of the projection operator, the number of such pixels, and the volume of the grid cell in Fourier space.

Alternatively, to estimate a band power for estimate for the power spectrum $P(k)$ from a pair of magnetic field components $(ij)$, we can use
\begin{equation}
\bar P_{(ij)}(k_1 <k < k_2) = \frac{\Delta_{{\rm vol}\,k}}{(2\pi)^3}  \frac{ \sum_{\mathbf{p}} {\rm Re}(B_i(\mathbf{p}) B^*_j(\mathbf{p})) }{ \sum_{\mathbf{p}} P_{ij}(\mathbf{p)}}
\end{equation}
where the sum is over Fourier pixels $p$ within the band for which the projection operator is not zero.  It is normalized by the number of such pixels and the volume of the grid cell in Fourier space.
The power estimates from the $(ij)$ components can be further averaged if desired.


\bibliography{ref}
\bibliographystyle{apsrev}

\end{document}
